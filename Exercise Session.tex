\documentclass[a4paper,10pt]{article}
\usepackage{graphicx} % Required for inserting images
\usepackage[english]{babel}
%4 stackanchor  
\usepackage{stackengine}
% define nice looking boxes
\usepackage[many]{tcolorbox}

% a base set, that is then customised
\tcbset {
  base/.style={
    boxrule=0mm,
    leftrule=1mm,
    left=1.75mm,
    arc=0mm, 
    fonttitle=\bfseries, 
    colbacktitle=black!10!white, 
    coltitle=black, 
    toptitle=0.75mm, 
    bottomtitle=0.25mm,
    title={#1}
  }
}
\definecolor{brandblue}{rgb}{0.34, 0.7, 1}
\newtcolorbox{mainbox}[1]{
  colframe=brandblue, 
  base={#1}
}

\definecolor{orange}{rgb}{1, 0.55, 0.3}
\newtcolorbox{tbox}[1]{
  colframe=orange, 
  base={#1}
}

\definecolor{green}{rgb}{0.294, 0.729, 0.254}
\newtcolorbox{bembox}[1]{
  colframe=green, 
  base={#1}
}

\definecolor{red}{rgb}{0.99, 0.04, 0.99}
\newtcolorbox{tipbox}[1]{
  colframe=red, 
  base={#1}
}

\newtcolorbox{defbox}[1]{
  colframe=black!20!white,
  base={#1}
}
% Mathematical typesetting & symbols
\usepackage{amsthm, mathtools, amssymb} 
\usepackage{marvosym, wasysym}


\allowdisplaybreaks

% Tables
\usepackage{tabularx, multirow}
\usepackage{booktabs}
\renewcommand*{\arraystretch}{2}

% Make enumerations more compact
\usepackage{enumitem}
\setitemize{itemsep=0.5pt}
\setenumerate{itemsep=0.75pt}

% To include sketches & PDFs
\usepackage{graphicx}

% For hyperlinks
\usepackage{hyperref}
\hypersetup{
  colorlinks=true
}
% Math helper stuff
\def\limn{\lim_{n\to \infty}}
\def\limxo{\lim_{x\to 0}}
\def\limxi{\lim_{x\to\infty}}
\def\limxn{\lim_{x\to-\infty}}
\def\sumk{\sum_{k=1}^\infty}
\def\sumn{\sum_{n=0}^\infty}
\def\R{\mathbb{R}}
\def\dx{\text{ d}x}
\usepackage[utf8]{inputenc}

\title{Exercise Sessions}
\author{Konstantin Lucny}
\date{HS 2023}

\begin{document}
\maketitle
\section{Class}
\subsection{Important Commandline commands}
\begin{itemize}
    \item \textbf{whoami}: prints the login name of the current user
    \item \textbf{pwd}: prints the working directory
    \item \textbf{ls}: lists files and directories
    \item \textbf{cd}: changes the current working directory to the given pathname (e.g. \textit{cd /home/username/ex1})
    \item \textbf{mkdir}: creates a directory (e.g. \textit{mkdir /home/username/ex1/newfolder})
    \item \textbf{rmdir}: removes a directory (only removes empty directories)
    \item \textbf{cp}: copies files/folders from one location to another (e.g. \textit{cp /etc/hosts /home/username})
    \item \textbf{mv}: move/rename existing files/folders (e.g. \textit{mv /home/username/hosts /home/username/ex1/newfolder})
    \item \textbf{rm}: removes files/folders (e.g. \textit{rm /home/username/ex1/newfolder/hosts})
    \item \textbf{ps}: see the processes associated with the current shell (\textit{ps –ef} to get a full listing of all processes in the system)
    \item \textbf{top}: display the processes using the most CPU time (Quit with \textit{q})
    \item \textbf{kill}: terminates a process (e.g. \textit{kill <ProcessID>}; \textit{-9} option to force kill)
    \item \textbf{gedit, emacs, vi/vim}: useful text editors for writing your programs and editing files
    \item \textbf{cat, more, less}: useful to view files
    \item \textbf{grep}: useful for searching text files
    \item \textbf{gcc/gdb}: compilers and debuggers
    

\end{itemize}

\end{document}
